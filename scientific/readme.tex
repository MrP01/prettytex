\documentclass{report}

\usepackage{preamble}   % should be loaded before math
\usepackage{math}       % requires preamble

\lstset{style=code_style} % sets the syntax highlighting of listings to code_style

\begin{document}
  \section{General Information}
    The files \textit{math.sty} and \textit{preamble.sty} should provide you a simple yet effectie suite of
    macros for quick writing of mathematical/scientific papers. To properly load them you should include the
    following in your preamble:
    % listings does not directly support writing latex-code in the environment
    \begin{center}
      \lstinputlisting[language={latex}]{preamble_listing.tex}  
    \end{center}

    It is important that you maintain the order of the packags, since math.sty uses some packages included in
    preamble.sty. Other than providing an extensive list of mathematical operators from math, there 
    are some useful commands in preamble.sty too.  The one that I myself use quite often is
    \lstinline[language=latex]|\col{<color>}{<text>}|. Altough xcolor defines \lstinline|\textcolor|,
    it can get kind of ''clunky'' in tables or similar, so i wrote a shorter command.


  \section{Symbol Index}
    % \lstinline cannot be used in \multicolumn
    % see (in german) https://golatex.de/tabelle-mit-multicolumn-und-lstinline-t20748.html
    % TODO: find better/more elegant solution

    \newsavebox\tveccol
    \begin{lrbox}{\tveccol}
      \begin{minipage}[t]{3cm}
        \lstinline|\tvec{x_1}{x_2}{x_3}|
      \end{minipage}
    \end{lrbox}

    \newsavebox\arsinhcol
    \begin{lrbox}{\arsinhcol}
      \begin{minipage}[t]{3cm}
        \lstinline|\Arsinh|
      \end{minipage}
    \end{lrbox}

    \newsavebox\arcoshcol
    \begin{lrbox}{\arcoshcol}
      \begin{minipage}[t]{3cm}
        \lstinline|\Arcosh|
      \end{minipage}
    \end{lrbox}

    \newsavebox\artanhcol
    \begin{lrbox}{\artanhcol}
      \begin{minipage}[t]{3cm}
        \lstinline|\Artanh|
      \end{minipage}
    \end{lrbox}

    \newsavebox\arcotanhcol
    \begin{lrbox}{\arcotanhcol}
      \begin{minipage}[t]{3cm}
        \lstinline|\Arcoth|
      \end{minipage}
    \end{lrbox}

    \newsavebox\arccotcol
    \begin{lrbox}{\arccotcol}
      \begin{minipage}[t]{3cm}
        \lstinline|\arccot|
      \end{minipage}
    \end{lrbox}

    

  \begin{center}
    \setlength\LTleft{.25cm}
    \setlength\LTright{.25cm}
    \begin{longtable}{Cl Cc Cc Cl Cc Cc}
      \textbf{Symbol} & \textbf{Math-Mode} & \textbf{Result} &  \textbf{Symbol} & \textbf{Math-Mode} & \textbf{Result}\\
      \multicolumn{2}{Cl}{\textbf{Symbol}} & \multicolumn{2}{Cc}{\textbf{Math-Mode}} & \multicolumn{2}{Cc}{\textbf{Result}}\\
      \hhline{======}
      \multicolumn{6}{Cc}{\textit{Vectors}}\\
      \hline
      Column Vector & \lstinline|\pvec{x_1}{x_2}|&$\pvec{x_1}{x_2}$ & 
        Dot-Product & \lstinline|\dotp{x_1}{x_2}| & $\dotp{x_1}{x_2}$\\
      \multicolumn{2}{Cl}{Column Vector} & \multicolumn{2}{Cc}{\usebox\tveccol} &
        \multicolumn{2}{Cc}{$\tvec{x_1}{x_2}{x_3}$} \\
      \hline
      \multicolumn{6}{Cc}{\textit{Matrices}}\\
      \hline
      Bold faced Matrix & \lstinline|\mat{M}| & $\mat{M}$&Determinant & \lstinline|\det|&$\det$\\
      Matrix-Rank (de) & \lstinline|\Rang| & $\Rang$ & Matrix-Rank (en) &\lstinline|\Rank|&$\Rank$\\
      Matrix-Trace (de) & \lstinline|\Spur| & $\Spur$ & Matrix-Trace (en) &\lstinline|\Trace|&$\Trace$\\
      Adjunct-Matrix & \lstinline|\Adj| & $\Adj$ & Cofactor-Matrix & \lstinline|\Cof| & $\Cof$\\
      Identity-Matrix (de) &\lstinline|\imate| & $\imate$ & Identity-Matrix (en) & \lstinline|\imati| & $\imati$\\
      \hline
      \multicolumn{6}{Cc}{\textit{Calculus and Functions}}\\
      \hline
      Differential d & \lstinline|\diff|& $\diff$ & Divergence & \lstinline|\divs| & $\divs$\\
      Derivative & \lstinline|\der{f}{x}| & $\der{f}{x}$ & 
        Partial Derivative & \lstinline|\per{f}{x_1}| & $\per{f}{x_1}$\\
      n-th Derivative & \lstinline|\ner{f}{x}{n}|&$\ner{f}{x}{n}$ &
        n-th Partial Derivative & \lstinline|\pnr{f}{x_1}{n}| & $\pnr{f}{x_1}{n}$\\
      Curl (de) & \lstinline|\rot| & $\rot$ & Curl (en) & \lstinline|\curl| & $\curl$\\
      Limit (noarg) & \lstinline|\lims| & $\lims$ & Limit & \lstinline|\lim{n}{\infty}| & $\lim{n}{\infty}$\\
      Infimum (noarg) & \lstinline|\infs| & $\infs$ & Infimum & \lstinline|\inf{M}| & $\inf{M}$\\
      Supremum (noarg) & \lstinline|\sups| & $\sups$ & Supremum & \lstinline|\sup{M}| & $\sup{M}$\\ 
      Limes Inferior (noarg) & \lstinline|\liminfs| & $\liminfs$ & Limes Inferior & \lstinline|\liminf{n}{\infty}| &
        $\liminf{n}{\infty}$\\
      Limes Superior  (noarg) & \lstinline|\limsups| & $\limsups$ & Limes Superior & $\limsup{n}{\infty}$ & 
        $\limsup{n}{\infty}$\\
      Function Image (de) & \lstinline|\Bild| & $\Bild$ & Function Image (en) & \lstinline|\Img| & $\Img$ \\
      \multicolumn{2}{Cl}{Area Sinus hyperbolicus} & \multicolumn{2}{Cc}{\usebox\arsinhcol} & 
        \multicolumn{2}{Cc}{$\Arsinh$}\\
      \multicolumn{2}{Cl}{Area Cosinus hyperbolicus} & \multicolumn{2}{Cc}{\usebox\arcoshcol} & 
        \multicolumn{2}{Cc}{$\Arcosh$}\\
      \multicolumn{2}{Cl}{Area Tangens hyperbolicus} & \multicolumn{2}{Cc}{\usebox\artanhcol} & 
        \multicolumn{2}{Cc}{$\Artanh$}\\
      \multicolumn{2}{Cl}{Area Cotanges hyperbolicus} & \multicolumn{2}{Cc}{\usebox\arcotanhcol} & 
        \multicolumn{2}{Cc}{$\Arcoth$}\\
      \multicolumn{2}{Cl}{Arcus Cotanges} & \multicolumn{2}{Cc}{\usebox\arccotcol} & 
        \multicolumn{2}{Cc}{$\arccot$}\\
      \pagebreak
      \hline
      \multicolumn{6}{Cc}{\textit{Logic}}\\
      \hline
      Bijunction & \lstinline|\bij| & $\bij$ \\
      Equivalent & \lstinline|\eqv| & $\eqv$ & Not Equivalent & \lstinline|\neqv| & $\neqv$\\
      Right Subjunction & \lstinline|\subj| & $\subj$ & Left Subjunction & \lstinline|\lsubj| & $\lsubj$\\
      Not Right Subjunction & \lstinline|\nsubj| & $\nsubj$ & Not Left Subjunction & \lstinline|\nlsubj| & $\nlsubj$\\
      Right Implication & \lstinline|\implies| & $\implies$ & Left Implication & \lstinline|\limplies| & $\limplies$\\
      Not Right Implication & \lstinline|\nimplies| & $\nimplies$ & Not Left Implication & \lstinline|\nlimplies| &
        $\nlimplies$\\
      Symbol for True (de) & \lstinline|\dtrue| & $\dtrue$ & Symbol for True (en) & \lstinline|\etrue| & $\etrue$\\
      Symbol for False (de) & \lstinline|\dfalse| & $\dfalse$ & Symbol for False (en) & \lstinline|\efalse| &$\efalse$\\
      \hline
      \multicolumn{6}{Cc}{\textit{Equations}}\\
      \hline
      Should be equal to & \lstinline|\feq| & $\feq$\\
      \hline
      \multicolumn{6}{Cc}{\textit{Constants}}\\
      \hline
      Imaginary Unit & \lstinline|\i| &  $\i$ & Jimaginary Unit (EE) & \lstinline|\j| & $\j$\\
      Euler's Number & \lstinline|\e| & $\e$\\
      \hline 
      \multicolumn{6}{Cc}{\textit{Number Theory}}\\
      \hline
      GCD (de) & \lstinline|\ggT| & $\ggT$ & GCD (en)&\lstinline|\gcd|& $\gcd$\\
      LCM (de) & \lstinline|\kgV| & $\kgV$ & LCM (en)&\lstinline|\lcm|& $\lcm$\\
      \hline
      \multicolumn{6}{Cc}{\textit{Signal Transforms}}\\
      \hline
      Laplace Transform & \lstinline|\ltr{x}| & $\ltr{x}$ & Z Transform & \lstinline|\ztr{x}| & $\ztr{x}$\\
      Laplace Transform & \lstinline|\lap{x}| & $\lap{x}$ & Inverse Laplace Transform & \lstinline|\ilap{x}| & 
        $\ilap{x}$\\
      Z-Transform & \lstinline|\zat{x}| & $\zat{x}$ & Inverse Z-Transform & \lstinline|\izat{x}| & $\izat{x}$\\
      Fourier Transform & \lstinline|\frt| & $\frt$ \\
      Fourier Transform & \lstinline|\fat{x}| & $\fat{x}$ & Inverse Fourier Transform & \lstinline|\ifat{x}| &
        $\ifat{x}$\\
      Fourier Series (de) & \lstinline|\frr| & $\frr$ & Fourier Series (en) & \lstinline|\frs| & $\frs$\\
      DFT & \lstinline|\dft| & $\dft$ & DTFT & \lstinline|\dtft| & $\dtft$\\
      \hline
      \multicolumn{6}{Cc}{\textit{Custom TikZ-Symbols for Signal Transforms}}\\
      \hline
      Laplace Transform & \lstinline|\ltransf| & $\ltransf$ & Inverse Laplace Transform & 
        \lstinline|\Ltransf| & $\Ltransf$\\
      Z Transform & \lstinline|\ztransf| & $\ztransf$ & Inverse Z Transform & \lstinline|\Ztransf| &
        $\Ztransf$\\
      \hline
      \multicolumn{6}{Cc}{\textit{Sets}}\\
      \hline
      Natural Numbers & \lstinline|\N| & $\N$ & Integers & \lstinline|\Z| & $\Z$\\
      Rational Numbers & \lstinline|\Q| & $\Q$ & Irrational Numbers & \lstinline|\I| & $\I$\\
      Real Numbers & \lstinline|\R| & $\R$ & Complex Numbers & \lstinline|\C|& $\C$\\
      Set of Primes & \lstinline|\P| & $\P$ & Transcendental Numbers & \lstinline|\T| & $\T$\\
      General Field (de) & \lstinline|\K| & $\K$ & General Field (en) & \lstinline|\F| & $\F$\\
      \hhline{======}\\
      \caption{All symbols and operators from math.sty}
    \end{longtable}
  \end{center}

  As you might have noticed, some of the entries in the table above feature either (de) or (en). These
  typically refer to language-dependet Operators. A classic example is the Curl of a Vector-Field. In English, the
  operator is either $\nabla\times\mat{V}$ or $\curl(\mat{V})$. In German however, the cross-prodcut
  $\nabla\times\mat{V}$ ist referred to as \textit{Rotation von $\mat{V}$}\footnote{Rotation of $\mat{V}$}. 
  Hence the Operator $\rot(\mat{V})$.

  There also exist some limits which take no arguments, which is listed with (noarg). This was mostly done
  to provide a simple text command for just the operator. If you e.g. just want to write: 
  \textit{The limes superior refers to the largest ...} and want to use the symbol $\limsups$ in text without any
  subscript.

  \section{A Word on Tables}
    Tables in \LaTeX can be quite a pain, especially correct vertical spacing and alignemnt. To avoid maximum 
    frustration, the package \lstinline|cellspace| is loaded. It allows to define a minimal distance to the top
    and the bottom of a row. To enable this functionality in your tables, you need to modify your column-list by
    adding \lstinline|S| in front of your column type, e.g. \lstinline|\begin{tabular}{Sc Sl Sr}|. 
    \textbf{Note:} If you have \lstinline|siunitx| loaded\footnote{preamble loads this package} you need to
    write \lstinline|Cc| instead.  

    The standard value for space to top/bottom is 4pt. You can change this by modifying the corresponging commands
    in preamble.sty:
    \begin{itemize}
      \item \lstinline|\setlength\cellspacetopline|  controls the spacing to the top
      \item \lstinline|\setlength\cellspacebottomline| controls the spacing to the bottom
    \end{itemize}

    preamble also includes the \lstinline|longtable| package. This allows for tables to perform pagebreak. A
    pagebreak can be manually inserted by typing \lstinline|\pagebreak| in the table-contents. In order for
    this to work, the longtable-environment mustn't be in a table-environment. So wrap your longtable in
    a center and put the caption as a row element. See \lstinline|readme.tex| for an example.

  \section{Authors Note}
    Since I am currently studying Information and Computer Engineering, I've only written macros for 
    corresponding fields (i.e. electrical engineering). So currently there are no neat macros for Chemistry or
    advanced Physics, etc. Since this repository is public you can Issue a feature request and given some 
    time, it should be implemented in a corresponding style.
\end{document}