\documentclass{article}

\usepackage{prettytex/base}
\usepackage{prettytex/math}

\title{Report title}
\author{Moritz \& Peter}
\date{27.11.2021}

\begin{document}
  \maketitle

  \tableofcontents
  \newpage

  \section{Aufgabenstellung}
  \section{Voraussetzungen und Grundlagen}
  \section{Versuchsanordnung}
  \section{Geräteliste}
  \begin{tabular}{|l|l|l|}
    \hline
    \textbf{Gerät} & \textbf{Hersteller}          & \textbf{Typ}                   \\
    \hline
    Stoppuhr       & Chung's Electronic Co., Ltd. & C-563                          \\ \hline
    Oszilloskop    & Hameg                        & Digital Storage Scope HM-205-2 \\ \hline
  \end{tabular}

  \section{Versuchsdurchführung und Messergebnisse}
  \section{Auswertung und Unsicherheitsanalyse}
  \section{Diskussion}
  \section{Zusammenfassung}

  \newpage
  \listoftables
  \listoffigures
\end{document}
