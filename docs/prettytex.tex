\documentclass{article}

% base prettytex 
\usepackage{prettytex/base}
\usepackage{prettytex/math}
\usepackage{prettytex/gfx}
\usepackage{prettytex/code}
\usepackage{prettytex/boxes}

% prettytex extensions
\usepackage{prettytex/math-theorems}
\usepackage{prettytex/gfx-circuits}
\usepackage{prettytex/gfx-components}

\fancyhf{}
\fancyhead[C]{\mintinline{latex}{prettytex} Documentation}
\fancyfoot[LE,RO]{\thepage}
\fancyfoot[LE,RO]{\leftmark}
\renewcommand{\headrulewidth}{.25pt}
\renewcommand{\footrulewidth}{.25pt}
\pagestyle{fancy}

\newcommand{\prettytex}{\mintinline{latex}{prettytex} }

\author{Moritz Mossböck}
\date{2022}
\title{Official \mintinline{latex}{prettytex} Documentation}

\begin{document}
  \pagenumbering{gobble}
  \maketitle
  \tableofcontents
  \newpage
  \pagenumbering{arabic}

  \section{About}
  The goal of \prettytex is to compress commonly used preambles into single style files, in order to provide a
  standardized preamble for your Latex-documents. However, not every invoice needs custom-signal-transforms symbols
  and a mathematical proof does not necessarily need styles for generating invoices, hence we modularized the
  preamble. The following (base) styles are currently available:
  \begin{itemize}
    \item \mintinline{latex}{base.sty}
    \item \mintinline{latex}{math.sty}
    \item \mintinline{latex}{gfx.sty}
    \item \mintinline{latex}{code.sty}
    \item \mintinline{latex}{boxes.sty}
    \item \mintinline{latex}{invoice.sty}
    \item \mintinline{latex}{contract.sty}
  \end{itemize}

  Additionally, some extensions are provided to extend base-functionality:
  \begin{itemize}
    \item \mintinline{latex}{math-theorems.sty}
    \item \mintinline{latex}{math-sigtrans.sty}
    \item \mintinline{latex}{gfx-components.sty}
    \item \mintinline{latex}{gfx-circuits.sty}
  \end{itemize}


  \section{Document Layout and Spacing}
  Document layout is specified in \verb|base.sty| with the package \verb|geometry|. We set the following options:
  \begin{minted}{latex}
\geometry{
  a4paper,
  twoside,
  left=2cm,
  right=2cm,
  top=1.25cm,
  bottom=1.25cm,
  includeheadfoot
}
\end{minted}

  As \verb|prettytex| was developed mainly for lab-reports, assignments and the likes, we set \verb|twoside| to
  account for possibly printing the finished document.


\end{document}
